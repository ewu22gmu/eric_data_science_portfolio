%--------------------------------------------------------------------+
% template_capstone_proposal_slide.tex                               |
% Copyright (C) 2024--2025 Dr. Kent L. Miller.  All rights reserved. |
% License: GNU GPL3+.  See http://www.gnu.org/licenses/              |
%--------------------------------------------------------------------+
\documentclass[clock,final,landscape,letterpaper]{slides}
\usepackage{graphicx,hyperref}
\usepackage{float}
\usepackage{tabularx}
\usepackage{cite}

\onlyslides{1-9999}

\makeatletter
\def\thebibliography#1{
  \list
  {[\arabic{enumi}]}{\settowidth\labelwidth{[#1]}\leftmargin\labelwidth
  \advance\leftmargin\labelsep
  \usecounter{enumi}}
  \def\newblock{\hskip .11em plus .33em minus .07em}
  \sloppy\clubpenalty4000\widowpenalty4000
  \sfcode`\.=1000\relax}
\let\endthebibliography=\endlist
\makeatother

\begin{document}
\title{CDS-492:  Capstone for Data Science \\ Fall 2025 \\ Presentation}
\author{Eric Wu}
\date{\today}
\maketitle

\addtime{60} % Slide 1: 1 minute
\begin{note}
  Start with an ``hook'' to capture the audience's interest.
\end{note}
\begin{slide}
  \textbf{Topic:}
  
  How powerful can a man's words be?

  During his second term, the 47th president of the United States, Donald Trump, has inflicted chaos and pandemonium on the markets via his enlightened posts on social media. This unease has caused retail and institutional investors to second-guess their risk tolerance, long-term holdings, and confidence in the continued financial growth of the United States.

  Do Donald Trump's colorful social media posts actually effect market performance in a measurable and meaningful manner the short and long-term?
\end{slide}

\addtime{60} % Slide 2: 1 minute
\begin{note}
  Explain the significance of this proposed project to science.
\end{note}
\begin{slide}
  \textbf{Significance:}

  The stock market affects all of us indirectly and directly for those with exposure in said markets. 
  Some of us depend on continued growth in the stock market for our savings or retirement, obtaining stable employment from publicly traded companies who raise capital from healthy stock markets, or rely on the growing stock market as an economic signal that the economy is thriving, bolstering consumer confidence, which in turn boosts consumer spending. 
  
\end{slide}

\addtime{60} % Slide 3: 1 minute
\begin{note}
  Describe prior research.
\end{note}
\begin{slide}
  \textbf{Prior research:}

  \begin{itemize}
    \item \textbf{Market Sentiment:} Sentiment betas are key for understanding stock market movement \cite{davidwugler2007} \cite{nofer2015}.
    \item \textbf{Data Sources:} Social media, online news, and forums are valuable sources for sentiment data \cite{johndavid2022} \cite{shahisahzulkernie2018} \cite{dassadhukhanchatterjeecharkabarti2022} \cite{Hutto_Gilbert_2014}.
    \item \textbf{Techniques:} 
    \begin{itemize}
        \item Dictionary/Lexicon-based models (e.g., VADER) are performant on news and social media posts \cite{johndavid2022} \cite{dassadhukhanchatterjeecharkabarti2022}.
        \item Topic modeling and joint sentiment-topic models are popular predictors of stock price movement \cite{NguyenThienHai2015Saos}.
    \end{itemize}
  \end{itemize}
\end{slide}

\addtime{60} % Slide 4: 1 minute
\begin{note}
  Describe the deliverables.
\end{note}
\begin{slide}
  \textbf{Deliverables:}

  \begin{itemize}
    \item Capstone project paper documenting my methodology, research, and findings
    \item Science paper documenting my findings
    \item Slide presentation for presenting my research and findings
  \end{itemize}
\end{slide}

\addtime{60} % Slide 5: 1 minute
\begin{note}
  Describe the experiment that produced the data set.
\end{note}
\begin{slide}
  \twocolumn \textbf{Method of inquiry:}

  I will be using the NADAQ and S\&P500 indicies' historic quotes from Yahoo Finance to approximate the overall effect of Donald Trump's social media posts on the broader market for ease of data compilation and a dataset of all of DJT's posts on Truth Social since his second term.
  
  \centering
  \includegraphics[width=0.5\linewidth]{yahoofinance_logo.png}

  \onecolumn
\end{slide}

\addtime{60} % Slide 6: 1 minute
\begin{note}
  Describe the data set that you will analyze.
\end{note}
\begin{slide}
  \textbf{Method of data collection:}
 
  {
  \footnotesize
  The first two datasets contain stock movement data for the following index funds, the SP500, Standards and Poors 500, and IXIC, the NASDAQ 500. 
  Yahoo fiance partners with trading exchanges and providers of financial market data to collect and provide historical stock quote data \cite{CSI_quote_20251022_IXIC} \cite{CSI_quote_20251022_GSPC}. 
  }

  Yahoo Finance Historical Quotes for the NASDAQ and S\&P500:
  \footnotesize\centering\begin{tabular}{lp{20cm}}
    \hline
    \emph{Column} & \emph{Comment} \\ 
    \hline
    Date            & Date \\
    Open            & Opening price of the equity \\
    High            & Maximum price of the equity for the given day\\
    Low             & Minimum price of the equity for the given day\\
    Close           & The closing price of the equity \\
  \end{tabular}
\end{slide}

\begin{slide}
  \textbf{Method of data collection (cont.):}

  {
  \footnotesize
  The third data set used for this project is a web scrape of all of Donald J. Trump's posts on his social media platform, Truth social. 
  This data includes his direct posts, re-posts, and his interactions with various truth social users. 
  This dataset contains the following features:
  }

  \footnotesize
  \centering
  \begin{tabular}{lp{20cm}}
    \hline
    \emph{Column} & \emph{Comment} \\ 
    \hline
    id            & The unique identifier for the post \\
    created at    & Timestamp when the post was made \\
    content         & The text content of the post \\
    url             & Direct link to the post on Truth Social \\
    media           & An array of image and video URLs if the post contains media \\
    replies count   & Number of replies to Trump post \\
    reblogs count   & Number of re-posts, or re-truths, to Trump post \\
    favourites count & Number of favorites to Trump post \\
    \hline
  \end{tabular}
\end{slide}

\addtime{60} % Slide 7: 1 minute
\begin{note}
  Describe how data will be analyzed
\end{note}
\begin{slide}
\footnotesize
  \textbf{Theory:}

The original hypothesis was that the sentiment of Donald Trump's posts would effect the stock market. 
I used index funds to abstract the stock market, in this case the S\&P500 and NASDAQ Composite. 
My null hypothesis for this initial theory was:
\begin{equation}
    H_{0} : \mu_{positive} = \mu_{neutral} = \mu_{negative}
\end{equation} 
My second hypothesis was that the topic of Donald Trump's posts would effect the stock market. 
I used index funds to abstract the stock market, in this case, I only used the S\&P500 index, because the performance of both indexes was similar over the timeframe of Trump's second presidency.
\begin{equation}
    H_{0} : \mu_{topic_i} = \mu_{topic_{i+1}} = \mu_{topic_{n}}
\end{equation} 

\end{slide}

\addtime{60} % Slide 8: 1 minute
\begin{note}
  Describe why this is a capstone.
\end{note}
\begin{slide}
  \textbf{Capstone:}

  The project integrates what I've learned in:
  \begin{itemize}
    \item CDS-230 Modeling and Simulation I
    \item CDS-301 Scientific Information and Data Visualization 
    \item CDS-321 Natural Language Processing
    \item CDS-402 Machine Learning
  \end{itemize}
    With the intention of discovering the power an individual's words can have.
\end{slide}

\addtime{60} % Slide 9: 1 minute
\begin{note}
  Result 1:
\end{note}
\begin{slide}
  \begin{center}
    {\textbf{Results: Sentiment Analysis Findings}}
  \end{center}
  %\vspace{0.2em}
  \begin{minipage}[c]{0.55\linewidth}
    \footnotesize % Use small font, not scriptsize, for readability
    \textbf{Objective:} Determine if the sentiment of DJT's posts affects stock market volatility (S\&P500 and NASDAQ).

    %\vspace{0.2em}
    \textbf{Null Hypothesis ($H_0$):} Mean volatility is equal across sentiment classes.
    \[ H_{0} : \mu_{positive} = \mu_{neutral} = \mu_{negative} \]

    \textbf{Findings:}
    \begin{itemize}
        \item S\&P500 p-value: $0.5216$
        \item NASDAQ p-value: $0.4274$
    \end{itemize}

    \vspace{1em}
    \textbf{Conclusion: We fail to reject the null hypothesis.} No statistically significant difference was found across sentiment classes.
  \end{minipage}
  \hfill % Pushes the minipages apart
  \begin{minipage}[c]{0.40\linewidth}
    \centering
    \includegraphics[width=\linewidth]{p11.png}
  \end{minipage}
\end{slide}

\addtime{60} % Slide 10: 1 minute
\begin{note}
  Result 2: LDA Topic Model
\end{note}
\begin{slide}
  \begin{center}
    {\textbf{\large Results: Topic Modeling Findings}}
  \end{center}
  \vspace{0.5em}
  \begin{minipage}[c]{0.55\linewidth}
    \small % Use small font, not scriptsize, for readability
    \textbf{Objective:} Determine if the topic of DJT's posts affects stock market volatility (S\&P500).

    \vspace{0.5em}
    \textbf{Null Hypothesis ($H_0$):} Mean volatility is equal across topic classes.
    \[ H_{0} : \mu_{topic_i} = \mu_{topic_{i+1}} = \mu_{topic_{n}} \]

    \textbf{Single Anova Findings:}
    \begin{itemize}
        \item S\&P500 p-value: $0.000272$
    \end{itemize}

    \vspace{1em}
    \textbf{Conclusion: We find evidence to reject the null hypothesis.} A statistically significant difference was found between topics.
  \end{minipage}
  \hfill % Pushes the minipages apart
  \begin{minipage}[c]{0.40\linewidth}
    \scriptsize
    \centering
    \begin{tabular}{lllll}
    \hline
    & diff & lwr & upr & pval \\ 
    \hline
    3-2 & 28.20 & 0.82 & 55.59 & 0.04 \\ 
    7-3 & -33.48 & -58.98 & -7.98 & 0.00 \\ 
    8-3 & -25.02 & -49.28 & -0.75 & 0.04 \\ 
    9-3 & -33.83 & -62.33 & -5.33 & 0.01 \\ 
   \hline
\end{tabular}
\vspace{0.5em}
  \raggedright
  \tiny
  \textbf{Post-Hoc Tukey LDA}

\footnote{The topics are descrived as: Topic 2: Negative posts; Topic 3: Trade War; Topic 7: Positive posts; Topic 8: War; Topic 9: Border control/imigration}

\end{minipage}
\end{slide}


\addtime{60} % Slide 11: 1 minute
\begin{note}
  Result 3: SeededLDA Topic Model
\end{note}
\begin{slide}
  \begin{center}
    {\textbf{\large Results: seededLDA Topic Modeling Findings}}
  \end{center}
  \vspace{0.5em}
  \begin{minipage}[c]{0.55\linewidth}
    \small % Use small font, not scriptsize, for readability
    \textbf{Objective:} Determine if the topic of DJT's posts affects stock market volatility (S\&P500).

    \vspace{0.5em}
    \textbf{Null Hypothesis ($H_0$):} Mean volatility is equal across topic classes.
    \[ H_{0} : \mu_{topic_i} = \mu_{topic_{i+1}} = \mu_{topic_{n}} \]

    \textbf{Single Anova Findings:}
    \begin{itemize}
        \item S\&P500 p-value: $0.00528$
    \end{itemize}

    \vspace{1em}
    \textbf{Conclusion: We find evidence to reject the null hypothesis.} A statistically significant difference was found between topics.
  \end{minipage}
  \hfill % Pushes the minipages apart
\begin{minipage}[c]{0.45\linewidth}
  \centering
  \tiny
  \begin{tabular}{lllll}
    \hline
      & diff & lwr & upr & pval \\ 
    \hline
    TRADE-other1 & 24.99 & 4.34 & 45.64 & 0.01 \\ 
    TRADE-other2 & 23.11 & 1.22 & 45.01 & 0.03 \\ 
    TRADE-other3 & 21.42 & 0.09 & 42.76 & 0.05 \\ 
    WAR-TRADE & -25.48 & -46.77 & -4.18 & 0.01 \\ 
     \hline
  \end{tabular}
  \vspace{0.5em}
  \raggedright
  \tiny
  \textbf{Post-Hoc Tukey seededLDA}
{
\tiny\footnote{The topics are descrived as: Trade: Posts about tariffs and trade wars; Other1: Negative Posts; Other2: Positive Nationalism; Other3: Digs at Democrats; War: Posts about actual war}
}
\end{minipage}
\end{slide}


\addtime{60} % Slide 12: 1 minute
\begin{note}
  Conclusions:
\end{note}
\begin{slide}
  \small
  \textbf{Conclusions:}

%\footnotesize
The tests I conducted on the use of sentiment to predict market movement were unexpectedly unfruitful. 
There where previous studies that found corelations between sentiment catagories extracted from forms of textual media, like articles and social media posts about the markets, however, due to my singular focus on an individual's posts on an arguably smaller and potentially more niche social media platform and due to the relience on pretrained and precomputed lexicon based sentiment analyzers, I was unable to reproduce the previous studies' results. 

This inability to reproduce the previous studies's results could also be down to a small dataset of about 3000 useable texts, of which were around 200 words or less. 
Another potential error may be due to the VADER lexicon having specifically been trained on written text compared to the tendency of social media users to type out what is more similar to what they may say in conversation. 
\end{slide}

\begin{slide}
  \small
\textbf{Conclusions (cont.)}

The lack of differences in the means of the daily volatility of both stock indexes, given the sentiment class, could be due to the VADER lexicon not performing well on my textual dataset, but it could also be due to the markets being more interested in the substance of specific DJT social media posts rather than the sentiment. 

These results from the topic models on the influence of the topic of DJT's social media posts, the substantive content of the posts, on the market were found to be measurable. 
Unfortnately I was unable to chain together the influence of the sentiment and the topic. 
If thete were to be more research on the subject of this research, I would recommend compiling a custom lexicon based sentiment analyzer based on DJT and other politicians or influential public figure's social media posts and their stated sentiments. 
I would also recommend exploring other topic models like top2vec or text embending extraction. 
\end{slide}

\addtime{60} % Slide 13: 1 minute
\begin{note}
  Disclose your source of data and any important reference materials.
\end{note}
\begin{slide}
  \textbf{Source Materials:}
  \begin{itemize}
    \item \url{https://www.marketwatch.com/investing/index/spx/download-data?startDate=1/1/2025&endDate=09/03/2025}
    \item \url{https://www.marketwatch.com/investing/index/comp/download-data?mod=mw_quote_tab}
    \item \url{https://github.com/stiles/trump-truth-social-archive/tree/main?tab=readme-ov-file}
  \end{itemize}
\end{slide}

\addtime{60} % Slide 14: 1 minute
\begin{note}
  Disclose your source of data and any important reference materials.
\end{note}
\begin{slide}
  \begin{center}
    \textbf{References}
  \end{center}
  \tiny 
  \begin{minipage}{0.9\textwidth} 
      \bibliographystyle{abbrv}
      \bibliography{gmuETD} 
  \end{minipage}
\end{slide}

\end{document}